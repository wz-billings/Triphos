\documentclass{article}

\usepackage{amsmath,amsthm}
\usepackage{graphicx}
\usepackage{amsfonts}

\usepackage{multicol}

\theoremstyle{plain}
\newtheorem{theorem}{Theorem}

\theoremstyle{definition}
\newtheorem*{definition}{Definition}
\newtheorem*{remark}{Remark}

% Commands
\newcommand{\Tri}{\mathbb{R}_T}
\newcommand{\degr}{^{\circ}}

\allowdisplaybreaks

\title{Algebra and geometry of Triphos}

\author{W. Zane Billings \  \& Andrew Penland}
\date{\today}

\begin{document}

	\maketitle

	\begin{abstract}
		Triphos is a number system motivated by the idea of using three axes
		(equally spaced \(120^{\circ}\) apart) on the two-dimensional
		coordinate plane rather than two. The Triphosian coordinate system has
		some similarities to the RGB additive coloring model of light. More
		importantly, Triphos has interesting algebraic and geometric
		properties. In this paper, we detail the construction of the Triphosian
		numbers, which has done previously, and we additionally provide a
		handful of new results about Triphos. Namely, regarding the algebraic
		structure of Triphos, we show that the field of Triphosian numbers has
		characteristic 0. We also provide novel results in Triphosian geometry:
		we construct the unit circle in Triphos, calculate the Triphosian
		equivalent of \(\pi\), and finally, we derive a form of the Pythagorean
		theorem in Triphos.
	\end{abstract}

	\section{Background}

	In Euclidean geometry, points in two-dimensional space are plotted on a
	plane with two axes (usually called \(x\) and \(y\)), placed
	perpendicularly (specifically, \(90\degr\) apart)--one axis specifying each
	dimension of the point. A consequence of this definition of the plane is
	that the plane can be subdivided into tiny squares. However, this
	definition of the plane is not the only definition allowing the
	visualization of two-dimensional points. Consider instead a world where the
	plane is defined by three axes, spaced \(120\degr\) apart. The world of
	Triphos is defined by these axes, usually called \(r, \ g,\) and \(b\). In
	this world, two-dimensional points are specified by three coordinates, one
	for each axis. One interesting consequence in this world is that we do not
	need negative numbers--one can reach any point on the plane from any other
	point by only moving in positive directions parallel to the three axes.

	There are currently two major examinations of the properties of Triphos.
	One paper focuses primarily on explicit derivations of the algebra of
	Triphos (\cite{egging}), while the other focuses on the motivation behind
	Triphos and many of its interesting properties (\cite{grossnickle}). The
	inspiration behind Triphos is the RGB model for additive mixing of
	light~\cite{grossnickle}. If equal quantities of red, blue, and green light
	are mixed, the result is white light, and this is exactly how addition in
	Triphos works. While the analogy to light begins to break down when we
	consider multiplication of Triphosian numbers, the world of Triphos still
	provides an interesting system to study.

	We now offer a brief recap of what these two papers have already found,
	along with the definitions which will be necessary for the remainder of the
	article.

	* definitions and equivalence class
	* addition and multiplication


	\section{Isomorphism to \(\mathbb{R}^2\)}

	* brief remark on field structure of triphos
	* show triphos has characteristic 0 and explain what this means
	* show isomorphism between triphos and R2. explain that this verifies our
	intuition--triphos is just another way to write down coordinates in the
	plane.

	\section{Isometric geometry of Triphos}

	* brief explanation of hexa-metric

	\section{Computer programs}

	\bibliographystyle{unsrt}
	\bibliography{refs.bib}

\end{document}
