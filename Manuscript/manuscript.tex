\documentclass{article}

\usepackage{amsmath,amsthm}
\usepackage{graphicx}
\usepackage{amsfonts}

\usepackage{multicol}

\theoremstyle{plain}
\newtheorem{theorem}{Theorem}

\theoremstyle{definition}
\newtheorem*{definition}{Definition}
\newtheorem*{remark}{Remark}

% Commands
\newcommand{\Tri}{\mathbb{R}_T}
\newcommand{\degr}{^{\circ}}

\allowdisplaybreaks

\title{Algebra and geometry of Triphos}

\author{W. Zane Billings \  \& Andrew Penland}
\date{\today}

\begin{document}

	\maketitle

	\begin{abstract}
		Triphos is a number system motivated by the idea of using three axes
		(equally spaced \(120^{\circ}\) apart) on the two-dimensional
		coordinate plane rather than two. The Triphosian coordinate system has
		some similarities to the RGB additive coloring model of light. More
		importantly, Triphos has interesting algebraic and geometric
		properties. In this paper, we detail the construction of the Triphosian
		numbers, which has done previously, and we additionally provide a
		handful of new results about Triphos. Namely, regarding the algebraic
		structure of Triphos, we show that the field of Triphosian numbers has
		characteristic 0. We also provide novel results in Triphosian geometry:
		we construct the unit circle in Triphos, calculate the Triphosian
		equivalent of \(\pi\), and finally, we derive a form of the Pythagorean
		theorem in Triphos.
	\end{abstract}

	\section{Background}

	In Euclidean geometry, points in two-dimensional space are plotted on a
	plane with two axes (usually called \(x\) and \(y\)), placed
	perpendicularly (specifically, \(90\degr\) apart)--one axis specifying each
	dimension of the point. A consequence of this definition of the plane is
	that the plane can be subdivided into tiny squares. However, this
	definition of the plane is not the only definition allowing the
	visualization of two-dimensional points. Consider instead a world where the
	plane is defined by three axes, spaced \(120\degr\) apart. The world of
	Triphos is defined by these axes, usually called \(r, \ g,\) and \(b\). In
	this world, two-dimensional points are specified by three coordinates, one
	for each axis. One interesting consequence in this world is that we do not
	need negative numbers--one can reach any point on the plane from any other
	point by only moving in positive directions parallel to the three axes.

	There are currently two major examinations of the properties of Triphos.
	One paper focuses primarily on explicit derivations of the algebra of
	Triphos (\cite{egging}), while the other focuses on the motivation behind
	Triphos and many of its interesting properties (\cite{grossnickle}). The
	inspiration behind Triphos is the RGB model for additive mixing of
	light~\cite{grossnickle}. If equal quantities of red, blue, and green light
	are mixed, the result is white light, and this is exactly how addition in
	Triphos works. While the analogy to light begins to break down when we
	consider multiplication of Triphosian numbers, the world of Triphos still
	provides an interesting system to study.

	We now offer a brief recap of what these two papers have already found,
	along with the definitions which will be necessary for the remainder of the
	article. We take the following definitions of Triphosian numbers with
	addition, multiplication, and the hexa-metric distance from
	\cite{grossnickle}.

	\begin{definition}[Triple relation]
		Let \(t = (r, g, b)\) be a triple with $r, g, b \in \mathbb{R}$ and
		also let \(c \in \mathbb{R}\). We define the ``triple relation'' as
		\((r_1, g_1, b_1) \sim (r_2, g_2, b_2)\) if \((r_2, g_2, b_2) = (r_1 +
		c, g_1 + c, b_1 + c\).
	\end{definition}

	This is an equivalence relation \cite{egging}.

	\begin{definition}[Triphosian numbers]
		The set of Triphosian real numbers, \(\mathbb{R}_T\), is the set of all
		equivalence classes of real-valued triples \(t = [(r, g, b)] \in
		\mathbb{R}^3\) under the triple relation. That is,
		\[\mathbb{R}_T = \mathbb{R}^3 / \sim.\]

		Addition of two Triphosian real numbers is defined as
		\[[(r_1, g_1, b_1)] + [(r_2, g_2, b_2)] = [(r_1 + r_2, g_1 + g_2, b_1 +
		b_2)].\]

		Multiplication of two Triphosian real numbers is defined as
		\[[(r_1, g_1, b_1)] \times [(r_2, g_2, b_2)] = [(r_1r_2 + g_1b_2 +
		b_1g_2, r_1g_2 + g_1r_2 + b_1b_2, r_1b_2 + g_1g_2 + b_1r_2)].\]

		The distance between two Triphosian real numbers is defined by the
		hexa-metric distance,
		\begin{align*}
			H([(r_1, g_1, b_1)], [(r_2, g_2, b_2)]) &=  \\
			\min\{ & |(r_1 - b_1) - (r_2 - b_2)| + |(g_1 - b_1) - (g_2 -
			b_2)|, \\
			& |(r_1 - g_1) - (r_2 - g_2)| + |(b_1 - g_1) - (b_2 - g_2)|, \\
			& |(g_1 - r_1) - (g_2 - r_2)| + |(b_1 - r_1) - (b_2 - r_2)| \}.
		\end{align*}
	\end{definition}

	Addition and multiplication with these definitions are well-defined,
	and the set of Triphosian numbers equipped with these two operations
	forms a field , and this field equipped with the hexa-metric distance is a
	metric space \cite{egging}.

	Finally, since the set of Triphosian numbers is a set of equivalence
	classes, it remains for us to discuss which member of each equivalence
	class is most useful for manipulation. Both \cite{egging} and
	\cite{grossnickle} define a ``standard'' or ``reduced'' form of a
	Triphosian number to have all nonnegative coordinates, with one coordinate
	being zero. (Indeed, \cite{grossnickle} defines Triphosian numbers as only
	having nonnegative coordinates, but this restriction was not imposed by
	\cite{egging}, where it is noted that this restriction is plausible but not
	necessary.) The transformation used by both previous works is
	\[t - \min\{r, g, b\} = [(r - \min\{r, g, b\}, g - \min\{r, g,
	b\}, b - \min\{r, g, b\})].\]
	While this definition of standard form is quite useful as it allows for
	easy determination of where a point should lie on the plane, it will be
	less useful for some of the proofs we discuss in the next section. So, we
	will refer to the previous definition as ``standard form,'' and we define
	the ``reduced form'' of a Triphosian number as follows.
	\begin{definition}[Reduced form]
		A Triphosian number $t$ is in reduced form if $b = 0$. Any Triphosian
		number can be converted to reduced form, which we denote $\bar{t}$ by
		the transformation
		\[\bar{t} = [(r - b, g - b, 0)].\]
	\end{definition}


	\section{Isomorphism to \(\mathbb{C}\)}

	Since Triphosian numbers are plotted on the two-dimensional plane, our
	intuition tells us that Triphosian numbers should, in some way, be
	equivalent to the Cartesian two-dimensional plane. However, the
	transformation
	which converts between a Triphosian number \(t = (r, g, b)\) and its
	Cartesian equivalent \(z = (x, y)\) given by \cite{egging},
	\[ (x, y) = (r, g, b) \begin{pmatrix} 1 & 0 \\ -\frac{1}{2} &
	\frac{\sqrt{3}}{2} \\ -\frac{1}{2} & -\frac{\sqrt{3}}{2} \end{pmatrix}, \]
	is not injective---multiple Triphosian numbers will be mapped to the
	same pair of Cartesian coordinates.

	However, by examining the equivalence classes of triples and making use of
	the standard form transformation we defined previously, we can construct a
	bijection between the Triphosian real numbers and $\mathbb{C}$. We note
	that we construct a bijection between $\Tri$ and $\mathbb{C}$ rather than
	$\Tri$ and $\mathbb{R}^2$ for an important reason: multiplication in
	Triphos obeys the same rules as multiplication of complex numbers (see the
	discussion in both \cite{egging} and \cite{grossnickle}). This suggests an
	isomorphism to $\mathbb{C}$.

	\begin{theorem}
		$\Tri$ is isomorphic to $\mathbb{C}$.
	\end{theorem}

	\begin{proof}
		Define the mapping \(\lambda: \Tri \to \mathbb{C}]\) such that
		\[\lambda([(r, g, b)]) = (r - b) + (g - b)i.\]

		First we will show that \(\lambda\) is a bijection.
		\begin{enumerate}
			\item The mapping \(\lambda\) is injective: to prove this, let
			$t_1, t_2 \in \Tri$ such that \(\lambda(t_1) = \lambda(t_2)\) but
			\(t_1 \neq t_2\).
		\end{enumerate}
	\end{proof}

	\section{Isometric geometry of Triphos}

	* brief explanation of hexa-metric

	\section{Computer programs}

	\bibliographystyle{unsrt}
	\bibliography{refs.bib}

\end{document}
