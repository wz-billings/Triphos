\documentclass[11pt]{article}

% Packages to use
\usepackage{amsthm,amssymb,amsmath}
\usepackage{mathtools}
\usepackage{microtype}
\usepackage{multicol}
\usepackage[margin=1in]{geometry}

% Title data
\title{\textit{Triphos}: Algebra and Geometry of an Isometric World}
\author{Zane Billings}
\date{}

% Commands
\newcommand{\Tri}{\mathbb{R}_T}
\newcommand{\degr}{^{\circ}}

% Theorem styles
\theoremstyle{definition}
\newtheorem{definition}{Definition}

\theoremstyle{plain}
\newtheorem{theorem}{Theorem}
\newtheorem{result}{Result}

\theoremstyle{remark}
\newtheorem*{remark}{Remark}
\newtheorem*{conjecture}{Conjecture}

\begin{document}

	\maketitle
	\begin{abstract}
		Triphos is a number system motivated by the idea of using three axes
		(equally spaced \(120^{\circ}\) apart) on the two-dimensional
		coordinate plane rather than two. The Triphosian coordinate system has
		some similarities to the RGB additive coloring model of light. More
		importantly, Triphos has interesting algebraic and geometric
		properties. In this paper, we detail the construction of the Triphosian
		numbers, which has done previously, and we additionally provide a
		handful of new results about Triphos. Namely, regarding the algebraic
		structure of Triphos, we show that the field of Triphosian numbers has
		characteristic 0. We also provide novel results in Triphosian geometry:
		we construct the unit circle in Triphos, calculate the Triphosian
		equivalent of \(\pi\), and finally, we derive a form of the Pythagorean
		theorem in Triphos.
	\end{abstract}

	\section{Introduction}

	In Euclidean geometry, points in two-dimensional space are plotted on a
	plane with two axes (usually called \(x\) and \(y\)), placed
	perpendicularly (specifically, \(90\degr\) apart)--one axis specifying each
	dimension of the point. A consequence of this definition of the plane is
	that the plane can be subdivided into tiny squares. However, this
	definition of the plane is not the only definition allowing the
	visualization of two-dimensional points.

	Consider instead a world where the plane is defined by three axes, spaced
	\(120\degr\) apart. The world of Triphos is defined by these axes, usually
	called \(r, \ g,\) and \(b\). In this world, two-dimensional points are
	specified by three coordinates, one for each axis. One interesting
	consequence in this world is that we do not need negative numbers--one can
	reach any point on the plane from any other point by only moving in
	positive directions parallel to the three axes. Examining
	Figure~\ref{fig:grid}, one of the most interesting and motivating
	consequences of this definition can be observed: if we place ourselves at
	the origin, denoted \((0,0,0)\), and move one unit along the red axis, one
	unit parallel to the green axis, and finally one unit parallel to the blue
	axis, we have ended up back at the origin! This observation, that \((1,1,1)
	= (0,0,0)\), and consequently, that \((c,c,c) = (0,0,0)\) for any real
	number \(c\), is the major motivation behind the study of Triphos.

	There are currently two major examinations of the properties of Triphos.
	One paper focuses primarily on explicit derivations of the algebra of
	Triphos (\cite{egging}), while the other focuses on the motivation behind
	Triphos and many of its interesting properties (\cite{grossnickle}). The
	inspiration behind Triphos is the RGB model for additive mixing of
	light~\cite{grossnickle}. If equal quantities of red, blue, and green light
	are mixed, the result is white light, and this is exactly how addition in
	Triphos works. While the analogy to light begins to break down when we
	consider multiplication of Triphosian numbers, the world of Triphos still
	provides an interesting system to study.

	Here, we will cover the formal definition of the Triphosian numbers,
	briefly show that Triphos is both a field and a metric space (all of these
	results were previously proven in the former literature
	(\cite{egging,grossnickle}), but they are worth repeating), before deriving
	several new results. We will show that the field of Triphosian real numbers
	has characteristic zero, as well as show some interesting results in
	Triphosian geometry: the Triphosian unit circle is a hexagon, the
	Triphosian equivalent of \(\pi\) is equal to three, and finally, we derive
	a version of the Pythagorean theorem which applies to Triphosian triangles.

	\begin{figure}[tbp]
		\centering
		\includegraphics[width=3in]{triphos_grid.png}
		\caption{This figure, from~\cite{egging}, is an excellent visualization
		of the three Triphosian axes.}
		\label{fig:grid}
	\end{figure}

	\section{Algebra of Triphos}

	We first concern ourselves with the formal algebraic definition of the
	Triphosian numbers. While the existing literature deals with Triphosian
	triples directly for the construction of Triphos, we define Triphos as a
	set of equivalence classes.

	\subsection{Definition of Triphos}

	Consider the set of all real-valued triples. We define the following
	equivalence relation on these triples, representing the observation that
	\((c,c,c)\) represents the origin for any \(c \in \mathbb{R}\).

	\begin{definition}{Triple relation~\cite{egging}.}
		Let \((r,g,b)\) be a triple, where \(r,g,b \in \mathbb{R}\). Also,
		suppose \(c \in \mathbb{R}\). We define the relation
		\((r_1,g_1,b_1)\sim(r_2,g_2,b_2)\) if \((r_2,g_2,b_2) = (r_1 + c, r_2 +
		c, r_3 + c)\).
	\end{definition}

	This relation is actually an equivalence relation on the set of real-valued
	triples, which lends itself well to the definition of the set of Triphosian
	numbers.

	\begin{definition}{Equivalence relation~\cite{enderton}.}
		An equivalence relation is any relation between two sets which
		preserves the reflexive, symmetric, and transitive properties. That is,
		given a relation, \(R\), we have that:
		\begin{enumerate}
			\item \(R\) is reflexive: \(xRx\) for all \(x\).
			\item \(R\) is symmetric: if \(xRy\), then \(yRx\).
			\item \(R\) is transitive: if \(xRy\) and \(yRz\), then \(xRz\).
		\end{enumerate}
	\end{definition}

	\begin{theorem}
		The relation defined by \((r,g,b)\sim(r+c,g+c,b+c), \ c \in
		\mathbb{R}\) is an equivalence relation.~\cite{egging}
	\end{theorem}

	\begin{proof}
		In order to show that \(\sim\) is an equivalence relation, we must show
		that it is reflexive, symmetric, and transitive. A proof is provided in
		\cite{egging}, but reproduced in greater detail here.
		\begin{enumerate}
			\item Reflexive: let \((r,g,b) \in \Tri.\) By the definition of
			\(\sim\), \((r,g,b) \sim (r + 0, g + 0, b + 0) = (r,g,b)\), so
			\((r,g,b) \sim (r,g,b)\) and the reflexive property holds.
			\item Symmetric: let \((r_1,g_1,b_1), (r_2,g_2,b_2) \in \Tri\), and
			suppose that \((r_1, g_1,b_1) \sim (r_2,g_2,b_2)\). By definition
			of \(\sim\), there exists some \(c \in \mathbb{R}\) such that
			\((r_2, g_2, b_2) = (r_1 + c, g_1 + c, b_1 +c)\). This implies that
			\((r_2, g_2, b_2) \sim (r_1 + c + (-c), g_1 + c + (-c), b_1 +c +
			(-c)) = (r_1,g_1,b_1).\) So \((r_2,g_2,b_2) \sim (r_1,g_1,b_1)\)
			and thus the symmetric property holds.
			\item Transitive: let \((r_1,g_1,b_1, (r_2,g_2,b_2), (r_3,g_3,b_3)
			\in \Tri\) such that \((r_1,g_1,b_1) \sim (r_2,g_2,b_2)\) and
			\((r_2,g_2,b_2) \sim (r_3,g_3,b_3)\). By definition of \(\sim\), we
			have that there exist \(c,d \in \mathbb{R}\) such that
			\((r_2,g_2,b_2) = (r_1 + c,g_1 + c,b_1 + c)\) and \((r_3,g_3,b_3) =
			(r_2 + d,g_2 + d,b_2 + d)\). So we have that \((r_2,g_2,b_2) \sim
			(r_2 + d,g_2 + d,b_2 + d) = (r_1 + c + d, g_1 + c + d, b_1 + c +
			d)\). Thus, \((r_1, g_1, b_1) \sim (r_1 + c + d, g_1 + c + d, b_1 +
			c + d) = (r_3, g_3, b_3).\) Thus, \((r_1,g_1,b_1) \sim
			(r_3,g_3,b_3)\) and the transitive property holds.
		\end{enumerate}
		Since the reflexive, symmetric, and transitive properties hold,
		\(\sim\) is an equivalence relation on the set of real-valued triples.
	\end{proof}

	This relation has an interesting geometric meaning in Triphos: any points
	which are equivalent under \(\sim\) will be plotted in the same location on
	the Triphosian plane. Given that \(\sim\) is an equivalence relation, we
	can construct equivalence classes for each triple: these are an easy way to
	``combine'' all of the triples which lie on the same point in Triphos into
	one compact notation.

	\begin{definition}{Equivalence classes of triples.}
		We define the equivalence class of a triple, \((r,g,b)\) under
		\(\sim\), denoted \([(r,g,b)]\), to be the set of all triples related
		to \((r,g,b)\). Formally,
		\[[(r,g,b)] = \{(x,y,z) \mid (x,y,z)\sim(r,g,b)\}.\]
	\end{definition}

	This is a specific case of the general definition of an equivalence
	class~\cite{enderton}. Now that we have a notion of Triphosian equivalence
	classes, we are finally ready to give a precise definition of the
	Triphosian real numbers. In the two major papers to date about the Triphos
	system (\cite{egging,grossnickle}), the set of Triphosian reals is
	considered as the set of all real-valued triples. I feel that this
	definition is confusing and as such have chosen to define the set of
	Triphosian real numbers as the set of all equivalence classes.

	\begin{definition}{Triphosian real numbers.} The set of Triphosian real
	numbers (or simply, the Triphosian reals), denoted \(\Tri\), is defined as
	the set of all equivalence classes defined by the relation \(\sim\) on the
	set of real-valued triples. That is, in the typical formal notation,
		\[\Tri = \mathbb{R}^3/\sim.\]
	\end{definition}

	Noting that if we have a set \(S\) and an equivalence relation \(R\), the
	equivalence classes under \(R\) form a partition of \(S\)~\cite{enderton},
	we can say that every real-valued triple is in one, and only one,
	equivalence class.

	Since each equivalence class contains an uncountable number of triples
	(each equivalence class contains as many triples as there are real numbers
	\(c\), from the definition of our classes), we are faced with a few
	important questions about Triphosian numbers. Firstly, which triple should
	we use to represent each equivalence class? And secondly, how can we
	quickly tell if two triples are in the same equivalence class without
	plotting them? Both of these questions can be answered by defining a
	standard form for Triphosian numbers.

	\begin{definition}{Standard form~\cite{egging}.} A Triphosian number
	\([(r,g,b)]\) is said to be in standard form if all entries are
	non-negative (\(r,g,b > 0\)), and at least one entry is zero (\(r = 0\), or
	\(b = 0\), or \(g = 0\)).
	\end{definition}

	Importantly, each standard form Triphosian number uniquely represents a
	point on the plane~\cite{grossnickle}. A consequence of this fact is that
	if two Triphosian numbers are both in standard form, they can only be
	equivalent if each of their components is equivalent.

	\begin{definition}{Triphosian equality.~\cite{egging}}
		Let \([(r_1,g_1,b_1)], [(r_2,g_2,b_2)]\) be two Triphosian numbers in
		standard form. We say that \([(r_1,g_1,b_1)] = [(r_2,g_2,b_2)]\) if and
		only if \(r_1 = r_2, g_1 = g_2\), and \(b_1 = b_2\).
	\end{definition}

	Now with a formal definition of the Triphosian reals in our toolbox, we can
	examine the algebra of Triphos: as luck and the spirits of mathematics
	would have it, we can define operations on the Triphosian reals that give
	them an incredibly similar structure to the numbers we are used to working
	with.

	\subsection{Field Structure of Triphos}

	In order to examine the algebraic structure of Triphos, we first define
	notions of addition and multiplication of Triphosian numbers. Addition on
	the Triphosian numbers is equivalent to addition over vectors in
	\(\mathbb{R}^3\).

	\begin{definition}{Triphosian addition~\cite{egging}.}
		The sum of two Triphosian numbers, \([(r_1, g_1, b_1)], [(r_2, g_2,
		b_2)]\) is defined as:
		\[[(r_1, g_1, b_1)] + [(r_2, g_2, b_2)] = [(r_1 + r_2, g_1 + g_2, b_1 +
		b_2)].\]
	\end{definition}

	\begin{theorem}
		Triphosian addition is well-defined (i.e. unambiguous) over the
		Triphosian reals.~\cite{egging}
	\end{theorem}

	While Triphosian multiplication has a messy formula for direct
	computations, this definition of multiplication satisfies a similar
	geometric property of complex multiplication: if two Triphosian numbers are
	plotted as vectors on the Triphosian plane, their angles (from the red
	axis) add, and their magnitudes multiply (using the Hexa-Metric distance,
	which is defined later).

	\begin{definition}{Triphosian multiplication.~\cite{egging}}
		The product of two Triphosian numbers, \([(r_1, g_1, b_1)] \cdot [(r_2,
		b_2, g_2)]\) is defined as:
		\[[(r_1, g_1, b_1)] \cdot [(r_2, b_2, g_2)] = [(r_1r_2 + g_1b_2 +
		b_1g_2, r_1g_2 + g_1r_2 + b_1b_2, r_1b_2 + g_1g_2 + b_1r_2)].\]
	\end{definition}

	\begin{theorem}
		Triphosian multiplication is well-defined over the Triphosian
		reals.~\cite{egging}
	\end{theorem}

	The proofs that Triphosian addition and multiplication are well-defined can
	be found in~\cite{egging}. Now, in order to discuss the properties of these
	operations, we must first define several types of algebraic structures.

	\begin{definition}{Group~\cite{lang}.}
		A group, \(G = (S,*)\), is a set \(S\) equipped with a binary operation
		\(*\) satisfying the following properties:
		\begin{enumerate}
			\item \(G\) is closed under \(*\): if \(x,y \in G\), then \(x*y \in
			S\).
			\item \(G\) is associative under \(*\): if \(x,y,z \in G\), then
			\(x*(y*z) = (x*y)*z\).
			\item \(G\) has an identity element, \(e\), under the operation:
			\(x*e = x = e*x\) for all \(x \in G\).
			\item For every \(x \in G\), there is an inverse element: if \(x
			\in G\), then \(y \in G\) such that \(x*y = e = y*x\).
		\end{enumerate}
	\end{definition}

	\begin{definition}{Abelian group~\cite{lang}.}
		A group, \(G(S,+)\), is said to be Abelian if \(G\) is commutative
		under \(+\); that is, if \(x,y \in G\), then \(x + y = y + x\).
	\end{definition}

	\begin{definition}{Ring~\cite{lang}.}
		A ring, \(R = (S,+,\cdot)\), is a set \(S\) equipped with two binary
		operations, typically called addition (+) and multiplication
		(\(\cdot\)) satisfying the following properties:
		\begin{enumerate}
			\item \(R\) forms an Abelian group equipped with addition; that is,
			\((S,+)\) is a group (note that in a ring, the additive identity is
			often written as 0).
			\item \(R\) is closed under multplication: if \(x,y \in R\), then
			\(x \cdot y \in R\).
			\item \(R\) contains a multiplicative identity, 1: if \(x \in R\),
			then \(1 \cdot x = x = x \cdot 1\).
			\item \(R\) is associative under multplication: if \(x,y,z \in R\),
			then \(x\cdot(y\cdot z) = (x\cdot y) \cdot z\).
			\item \(R\) obeys the distributive property with respect to \(+\)
			and \(\cdot\); that is, given \(x,y,z \in R\) we have that \(x
			\cdot (y + z) = x\cdot y + x \cdot z\).
		\end{enumerate}
	\end{definition}

	\begin{definition}{Field~\cite{lang}.}
		A field, \(F = (S,+,\cdot)\), is a set \(S\) with two operations,
		typically called addition (\(+\)) and multiplication (\(\cdot\)),
		satisfying the following properties:
		\begin{enumerate}
			\item \(F\) is an Abelian group under \(+\).
			\item \(F - \{0\}\) is an Abelian group under \(\cdot\) (i.e.
			division by zero is not allowed; the additive identity is excluded
			as it has no multiplicative inverse).
			\item Addition distributes over multiplication in \(F\): if \(x,y,z
			\in F\), then \(x\cdot(y + z) = x\cdot y + x\cdot z\).
		\end{enumerate}
		This definition is equivalent to the 11 field axioms listed in many
		sources. Finally, note that every field is also a ring by definition.
	\end{definition}

	These definitions serve to illustrate the properties of the Triphosian
	reals, as well as provide the Triphosian reals with a common structure: by
	showing that the Triphosian reals have a well-studied structure, we
	immediately become aware of several facts that are true about the
	Triphosian reals.

	\begin{theorem}
		The set of Triphosian reals, equipped with Triphosian addition and
		multiplication, \((\Tri, +, \cdot)\), forms a field.~\cite{egging}
	\end{theorem}
	\begin{proof}
		In order to show that the Triphosian reals are a field, we will first
		show that the Triphosian reals, equipped with Triphosian addition, form
		an Abelian group. The proof of all of these statements can be found
		in~\cite{egging}.
		\begin{itemize}
			\item The Triphosian reals are closed under addition.
			\item The associative and commutative properties hold for
			Triphosian reals under addition.
			\item The additive identity is \([(0,0,0)]\).
			\item Let \([(r,g,b)] \in \Tri\). The additive inverse of
			\([(r,g,b)]\) is \([(-r,-g,-b)] \in \Tri\). When written in
			standard form this is \([(g+b, r+b, r+g)]\)~\cite{grossnickle}.
		\end{itemize}

		Furthermore, the following properties hold for Triphosian
		multiplication~\cite{egging}.
		\begin{itemize}
			\item The Triphosian reals are closed under multiplication.
			\item The associative and commutative properties hold for
			Triphosian reals under multiplication.
			\item The multiplicative identity is \([(1,0,0)]\).
		\end{itemize}

		Additionally the multiplicative inverse is calculated as
		\([(r/n,g/n,b/n)]\) where \(n = r^2 + b^2 + g^2 - rb - rg -gb\); this
		is not defined for \([(0,0,0)]\), but it is defined everywhere
		else~\cite{egging}. So, the above properties are sufficient to show
		that the set \(\Tri - \{[(0,0,0)]\}\) is an Abelian group under
		Triphosian multiplication.

		Finally, the distributive property is verified in~\cite{egging} as
		well. So, the set of Triphosian reals equipped with Triphosian addition
		and multiplication forms a field.
	\end{proof}

	With this knowledge in mind, our next goal is to define a notion of
	distance in Triphos. A way to measure distance is crucial for our results
	in the geometry of Triphos, so our next goal is to show that distance can
	be measured with all of the familiar properties of Euclidean distance in
	Triphos.

	\subsection{The Hexa-Metric}

	The distance in Triphos is known as the Hexa-Metric function. The
	Hexa-Metric function is similar to the Taxicab metric function in Euclidean
	space, as it only allows movement parallel to the axes. Interestingly,
	there is no unique shortest path between points in Triphos: between two
	points there can be many paths, which stay parallel to the axes, yet all
	have the same minimum distance. Visual demonstrations of this metric can be
	found in~\cite{egging}, just like many other results covered here.

	\begin{definition}{Hexa-Metric function~\cite{egging}.} The Hexa-Metric
	function is defined as \(H: \Tri \times \Tri \to \mathbb{R}\) such that
		\begin{align*}
			H([(r_1,g_1,b_1)], [(r_2,g_2,b_2)]) = \min{}\{ &|(r_1 - b_1) - (r_2
			- b_2)| + |(g_1 - b_1) - (g_2 - b_2)|, \\
			&|(r_1 - g_1) - (r_2 - g_2)| + |(b_1 - g_1) - (b_2 - g_2)|, \\
			&|(g_1 - r_1) - (g_2 - r_2)| + |(b_1 - r_1) - (b_2 - r_2)| \}.
		\end{align*}
	\end{definition}

	In addition to forming a field, if we equip the Triphosian reals with the
	Hexa-Metric function, we obtain another well-studied structure known as a
	metric space.

	\begin{definition}{Metric space~\cite{kaplansky}.}
		A metric space, \(M = (S,d)\), is a set \(S\) with a function \(d: S
		\times S \to \mathbb{R}\) defined for all \(a,b \in S\) such that:
		\begin{enumerate}
			\item \(d(a,a) = 0\) for all \(a\).
			\item \(d(a,b) > 0\) for all \(a \neq b\).
			\item \(d(a,b) = d(b,a)\).
			\item \(d(a,c) \leq d(a,b) + d(b,c)\) (the triangle inequality
			holds).
		\end{enumerate}
		Additionally, the function \(d\) of a metric space is called a metric.
	\end{definition}

	Intuitively, a metric space is a coordinate space where distance behaves
	the way we are used to--Euclidean space (the set of real-valued doubles
	equipped with the Euclidean distance function) is a metric space as well,
	and the notion of a generalized metric space generalizes many of the
	properties we know and love.

	\begin{theorem}
		The Triphosian reals equipped with the Hexa-Metric function, \((\Tri,
		H)\), forms a metric space.~\cite{egging}
	\end{theorem}

	We again cite the proof in~\cite{egging} for this result. Now, knowing that
	Triphos is a metric space, we have a notion of distance between Triphosian
	numbers--this result is important to the geometry of Triphos, and we will
	use the Hexa-Metric function to prove a handful of results in geometry.

	\section{Results}

	\subsection{Characteristic of the Field of Triphosian Reals}

	The first result we will derive is the characteristic of the field of
	Triphosian reals. In order to define the characteristic of a ring, we need
	the following definitions from algebra.

	\begin{definition}{Subgroup~\cite{lang}.}
		A subgroup, \(H\), of a group \(G = (S,*)\), is a subset of \(G\) which
		is a group under \(*\) itself.
	\end{definition}

	\begin{definition}{Subring~\cite{lang}.}
		A subring \(A\) of a ring \(R = (S, +, \cdot)\) is a subset of \(R\)
		which is a ring under \(+\) and \(\cdot\) itself. By definition, any
		subring \(A\) of \(R\) must be a subgroup of \(R\) under \(+\) as well.
	\end{definition}

	\begin{definition}{Ring homomorphism~\cite{lang}.}
		Let \(A\) and \(B\) be rings, and define a mapping \(f: A \to B\).
		Furthermore, let 0 represent the additive identity and let 1 represent
		the multiplicative identity. We say \(f\) is a ring homomorphism if
		\(f\) satisfies the following properties for all \(a_1, a_2 \in A\).
		\begin{multicols}{2}
			\begin{itemize}
				\item \(f(a_1 + a_2) = f(a_1) + f(a_2)\).
				\item \(f(0) = 0\).
				\item \(f(a_1 \cdot a_2) = f(a_1)\cdot f(a_2)\).
				\item \(f(1) = 1\).
			\end{itemize}
		\end{multicols}
	\end{definition}

	\begin{definition}{Isomorphism~\cite{lang}.}
		Any homomorphism \(f\) is an isomorphism if and only if \(f\) is also a
		bijection. Specifically, an isomorphism which is a group homomorphism
		is called a group isomorphism, an isomorphism which is a ring
		homomorphism is called a ring isomorphism, etc.
	\end{definition}

	\begin{definition}{Characteristic of a ring~\cite{lang}.}
		The characteristic of a ring, \(R = (S, +, \cdot)\), denoted
		\(\mathrm{char}(R)\), is the minimum positive integer \(n > 0\) such
		that there is a subring \(A\) of \(R\) which is isomorphic to
		\(\mathbb{Z}/n\mathbb{Z}\). If \(R\) has a subring which is isomorphic
		to \(\mathbb{Z}\), then \(R\) has characteristic zero.
	\end{definition}

	Now that we have the characteristic of a ring defined, and all of the
	definitions we need for our proof, we can state and prove the result.

	\begin{result}
		The field of Triphosian reals, \(\Tri\), has characteristic zero.
	\end{result}
	\begin{proof}
		Consider the Triphosian reals, \(\Tri\). In order to show that \(\Tri\)
		has characteristic 0, we will construct a subring of \(\Tri\) and show
		it is isomorphic to \(\mathbb{Z}\).

		Let \(S\) be a subset of \(\Tri\) such that \(S = \{[(k,0,0)]\mid k \in
		\mathbb{Z}\}\). First, we will show that \(S\) is an additive subgroup
		of \(\Tri\).
		\begin{enumerate}
			\item Closure: Let \([(m,0,0)],[(n,0,0)] \in S\). Then, \([(m,0,0)]
			+ [(n,0,0)] = [(m+n,0,0)]\). Since \(m+n \in \mathbb{Z}\), we have
			that \([(m+n,0,0)] \in S\), so \(S\) is closed under addition.
			\item Associativity: Associativity and commutativity are true for
			all elements of \(\Tri\), so both hold in \(S\).
			\item Identity: Since \(0 \in \mathbb{Z}\), \([(0,0,0)] \in S\),
			thus \(S\) contains the additive identity of \(R\).
			\item Inverse: Given \(x \in S\), we have \(y \in S\) such that \(x
			+ y = [(0,0,0)]\). Let \(x = [(k,0,0)] \in S\). Then the additive
			inverse of \(x\) is \(y = [(-k,0,0)] \in S\) (we see that \(x + y =
			[(k + (-k),0,0)] = [(0,0,0)]\)), so \(S\) contains an additive
			inverse of every element.
		\end{enumerate}

		This proves that \(S\) is a subgroup under addition, so it remains to
		show that \(S\) is also a ring.
		\begin{enumerate}
			\item Identity: since \(1 \in \mathbb{Z}\), we have that
			\([(1,0,0)] \in S\) and thus \(S\) contains the multiplicative
			identity of \(R\).
			\item Associativity and distributivity: the associative property
			under \(\cdot\) and the distributive property are inherited from
			\(R\).
			\item Closure: let \([(m,0,0)], [(n,0,0)] \in S\). Then,
			\([(m,0,0)]\cdot[(n,0,0)] = [(m\cdot n + 0 + 0, m\cdot 0 + 0\cdot n
			+ 0, m \cdot 0 + 0 + 0 \cdot n)] = [(m\cdot n,0,0)]\). Since \(m
			\cdot n \in \mathbb{Z}\), we have that \([(m\cdot n,0,0)] \in S\),
			so \(S\) is closed under multiplication.
		\end{enumerate}
		So, \(S\) is a ring, and in particular, a subring of \(\Tri\).

		Now we will define a mapping \(\lambda: \mathbb{Z} \to S\) and prove
		that this mapping is an isomorphism. Let \(\lambda: \mathbb{Z} \to S\)
		be defined by \(\lambda(n) = [(n,0,0)]\).

		First we will show that \(\lambda\) is a bijection.
		\begin{enumerate}
			\item The mapping \(\lambda\) is injective: to prove this, let
			\(m,n \in \mathbb{Z}\) such that \(f(m) = f(n)\) but \(m \neq n\).
			Then, \(\lambda(m) = [(m,0,0)] = [(n,0,0)] = \lambda(n)\). But
			since both of these Triphosian numbers are in standard form, for
			this to be true we must have that \(m = n\), which is a
			contradiction. Hence, if \(f(m) = f(n)\), then \(m=n\) and
			\(\lambda\) is an injection.
			\item The mapping \(\lambda\) is surjective: let \(y =[(x,0,0)] \in
			S\). We will show that there exists some \(x \in \mathbb{Z}\) such
			that \(f(x) = y\). Let \(x \in \mathbb{Z}\). Then, \(f(x) =
			[(x,0,0)]\) and hence \(f(x) = y\). So, \(\lambda\) is surjective.
		\end{enumerate}

		So, \(\lambda\) is a bijection. Now we will show that \(\lambda\) is a
		ring homomorphism.
		\begin{enumerate}
			\item The mapping \(\lambda\) correctly maps identities:
			\(\lambda(0) = [(0,0,0)]\) and \(\lambda(1) = [(1,0,0)]\) by
			definition.
			\item The mapping \(\lambda\) preserves addition: let \(m,n \in
			\mathbb{Z}\). We have that \(\lambda(n+m) = [(m+n,0,0)] = [(m,0,0)]
			+ [(n,0,0)] = \lambda(m) + \lambda(n)\), so addition is preserved.
			\item The mapping \(\lambda\) preserves multiplication: let \(m, n
			\in \mathbb{Z}\). Then, \(\lambda(m \cdot n) = [(m\cdot n, 0,0)].\)
			We have that \([(m,0,0)] \cdot [(n,0,0)] = [(m\cdot n + 0 + 0,
			m\cdot 0 + 0\cdot n + 0, m \cdot 0 + 0 + 0 \cdot n)] = [(m\cdot
			n,0,0)]\), so we see that \(\lambda(m\cdot n) = [(m,0,0)]\cdot
			[(n,0,0)] = \lambda(m) \cdot \lambda(n)\). So, multiplication is
			preserved.
		\end{enumerate}

		Thus, \(\lambda\) is a ring homomorphism. Since \(\lambda\) is both a
		ring homomorphism and a bijection, \(\lambda\) is a ring isomorphism.
		To recap, we have that \(S\) is a subring of \(\Tri\), and we have
		established that \(\lambda\) is a ring isomorphism between \(S\) and
		\(\mathbb{Z}\).

		This establishes that \(\Tri\) has characteristic 0.
	\end{proof}

	\subsection{The Triphosian Unit Circle and Pi}

	\begin{result}
		The Triphosian pi, \(\pi_T\), defined as the ratio of the circumference
		of a circle to its diameter in Triphos, using the Hexa-Metric distance,
		is \(\pi_T = 3\).
	\end{result}

	In order to calculate the Triphosian equivalent of \(\pi\), which we will
	denote \(\pi_T\), we must first construct a circle, specifically the unit
	circle.

	\begin{definition}{Unit circle.}
		The unit circle is the set of all points which are distance 1 away from
		the origin.
	\end{definition}

	So, in order to construct the Triphosian unit circle we define the set of
	points

	\[\{[(r,g,b)] \mid H([(r,g,b)], [(0,0,0)]) = 1\},\]

	where \(H\) is the Hexa-Metric function. A visual plot of the Triphosian
	unit circle can be seen in Figure \ref{fig:uc}.

	\begin{figure}[htbp]
		\centering
		\includegraphics[width=3in]{unitcircle.png}
		\caption{The Triphosian unit circle, generated in \texttt{MATLAB}.}
		\label{fig:uc}
	\end{figure}

	The unit circle in Triphos forms a hexagon, whose vertices are the set
	\[ \{[(1,0,0)], [(0,1,1)], [(0,1,0)], [(1,0,1)], [(0,0,1)], [(1,1,0)]\}. \]

	Verifying that these points are in the set requires one calculation using
	the Hexa-Metric function as described in the definition of the set above,
	and so it is not included here. Now, given the set of vertex points of the
	hexagon, we can compute the side lengths of the hexagon. We show the
	calculation of the length of the side formed by the points \([(1,0,0)]\)
	and \([(1,1,0)]\) below.
	\begin{align*}
		H([(1,0,0)], [(1,0,1)])
		&= \min\{
		\begin{aligned}[t]
			&|(1 - 0) - (1 - 1)| + |(0 - 0) - (0 - 1)|, \\
			&|(1 - 0) - (1 - 0)| + |(0 - 0) - (1 - 0)|, \\
			&|(0 - 1) - (0 - 1)| + |(0 - 1) - (1 - 1)| \}
		\end{aligned}\\
		&= \min\{2,1,1\} = 1
	\end{align*}

	Note that the length of each side is the same (we will not show each of the
	computations, as they are all equivalent to the computation above), so the
	``circumference'', or more accurately, the perimeter, of our unit circle,
	is \(6 \cdot 1 = 6\).

	Now, by visual inspection of Figure \ref{fig:uc}, we see that the diameter
	of this regular hexagon can be found by taking the distance between
	\([(1,0,0)]\) and \([(0,1,1)]\). (Or, equivalently, we could take the
	distance between \([(0,1,0)]\) and \([(1,0,1)]\) or between \([(0,0,1)]\)
	and \([(1,1,0)]\), but all of these lengths are equal.)
	\begin{align*}
		H([(1,0,0)], [(0,1,1)]) &= \min\{
		\begin{aligned}[t]
			&|(1 - 0) - (0 - 1)| + |(0 - 0) - (1 - 1)|, \\
			&|(1 - 0) - (0 - 1)| + |(0 - 0) - (1 - 1)|, \\
			&|(0 - 1) - (1 - 0)| + |(0 - 1) - (1 - 0)| \}
		\end{aligned} \\
		&= \min{(2,2,4)} = 2.
	\end{align*}

	So the perimeter of the unit circle in Triphos is \(6\) and the diameter is
	\(2\), and thus the Triphosian equivalent of \(\pi\) is \(\pi_T = 6/2 = 3\).


	\subsection{A Triphosian Pythagorean Theorem}

	\begin{result}
		In Triphosian coordinates, if a triangle with a \(60^{\circ}\) angle
		has leg lengths \(a\) and \(b\), then the length, \(c\) of the third
		side (which is opposite from a \(60^{\circ}\) angle) is \[c = \min
		(\left| a-b\right| +\left| a\right| ,\left| a\right| +\left| b\right|
		,\left| a-b\right| +\left|
		b\right| ).\]
	\end{result}

	Assume we have a triangle with a \(60^{\circ}\) angle. First, we will
	impose Triphosian coordinates onto the triangle. Place the origin of the
	coordinate system at the vertex of the \(60^{\circ}\) angle, as shown in
	Figure \ref{fig:tri}.

	\begin{figure}[htbp]
		\centering
		\includegraphics[width=3in]{tri2.png}
		\caption{A triangle with a \(60^{\circ}\): the origin is imposed on the
		vertex of the \(60^{\circ}\) angle. Note that once the origin is
		imposed, the triangle can be rotated to a similar position as this
		triangle without changing any of its properties.}
		\label{fig:tri}
	\end{figure}

	Now, we can assign a length to each of the legs (the sides forming the
	\(60^{\circ}\) angle). We assign one leg the arbitrary length \(a\) and the
	other leg the arbitrary length \(b\), as shown in Figure \ref{fig:tri_c}.
	Since we have imposed the origin of the coordinate system on our triangle
	in a specific way, we now know the three points that define the triangle.
	Note that one of the legs is aligned with the blue axis, and one of the
	legs is aligned with the red axis.

	Without loss of generality, we can assign the arbitrary length \(a\) to the
	side aligned with the blue axis. Note (as can be seen in either Figure
	\ref{fig:tri} or \ref{fig:tri_c}) that this side goes in the negative
	direction along the blue axis, so we can say that this leg of the triangle
	terminates at the point \([(0,0,-a)]\), which when converted to standard
	form is \([(a,a,0)]\).

	Similarly, we notice that the other leg is aligned in the positive
	direction on the red axis. We can assign this leg the arbitrary length
	\(b\), and then this leg terminates at the point \([(b,0,0)]\).

	Then, we can calculate the length of the remaining side, \(c\), as
	\[c = H([(a,a,0)], [(b,0,0)]),\]
	using the Hexa-Metric function.

	\begin{figure}[htbp]
		\centering
		\includegraphics[width=3in]{tri3.png}
		\caption{Because of the way we have imposed coordinates, we can use the
		arbitrary lengths of the legs to determine the coordinates which define
		the third side of the triangle. Then, the length of the third side is
		the length between these two points.}
		\label{fig:tri_c}
	\end{figure}

	We then compute the length \(c\) as
	\begin{align*}
		H([(a,a,0)], [(b,0,0)]) = \min{}\{ &|(a - 0) - (b - 0)| + |(a - 0) - (0
		- 0)|, \\
		&|(a - a) - (b - 0)| + |(0 - a) - (0 - 0)|, \\
		&|(a - a) - (0 - b)| + |(0 - a) - (0 - b)| \},
	\end{align*}

	which simplifies to

	\[c = \min{ (\left| a-b\right| +\left| a\right| ,\left| a\right| +\left|
	b\right| ,\left| a-b\right| +\left|
		b\right| )}.\]

	\subsection{Computer Codes}

	In order to derive many of these results, a suite of \texttt{MATLAB}
	functions was created to do math in Triphos. While this suite of functions
	is currently relatively small and needs to be expanded, it is still novel
	in that no function suite currently exists for Triphosian math. We hope to
	expand this suite of functions in \texttt{MATLAB}, or potentially port the
	functions into Python and expand the suite there.

	Finally, when the Triphosian Pythagorean theorem was explored, a
	Mathematica notebook was created and used to verify this geometric theorem,
	and to create a more complex version of the theorem defined in terms of
	Triphosian coordinates.

	\section{Conclusion and Future Work}

	In addition to the results we have defined, we have a set of conjectures
	about the algebraic structure of Triphos.

	\begin{conjecture}
		The field of Triphosian reals, \(\Tri\), contains subfields isomorphic
		to \(\mathbb{Q}\) and \(\mathbb{R}\).
	\end{conjecture}

	We believe that this conjecture can be proven by using a similar mapping to
	the mapping \(\lambda\) we defined previously.

	\begin{conjecture}
		The field of Triphosian reals contains a subgroup isomorphic to
		\(\mathbb{Z}^3/\sim\) (potentially a subring), and a subgroup
		isomorphic to \(\mathbb{Q}^3/\sim\) (potentially a subfield).
	\end{conjecture}

	\begin{conjecture}
		There is an isomorphism between \(\Tri\) and \(\mathbb{R}^2.\)
	\end{conjecture}

	In addition to showing these facts about the algebraic structure of
	Triphos, we are additionally interested in defining a further notion of
	trigonometry in Triphos. Our first goal in this regard would be to define a
	notion of trigonometric functions on Triphosian numbers.

	Finally, we have not presented some of the additional interesting results
	seen in~\cite{grossnickle}--notably, notions of Triphosian integers and
	primes, and a different place value system which may be more appropriate
	for describing Triphosian numbers. We are also interested in exploring
	properties and distribution of Triphosian primes, as well as other
	sequences of Triphosian integers.

	In conclusion, we have constructed the Triphosian real numbers, and defined
	their algebraic structure and notion of distance. Furthermore, as unique
	results, we have proven that the field of Triphosian integers has
	characteristic zero, and regarding Triphosian geometry, we have constructed
	the Triphosian unit circle and shown that \(\pi_T = 3\). We have also found
	an analog of the Pythagorean theorem in Triphosian geometry, and defined
	several more interesting problems in Triphos.

	\bibliographystyle{unsrt}
	\bibliography{ref}

\end{document}
